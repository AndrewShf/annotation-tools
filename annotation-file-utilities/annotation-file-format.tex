\documentclass{article}
\usepackage{fancyvrb}
\usepackage{fullpage}
\usepackage{relsize}
\usepackage{url}
\usepackage{hevea}
% \usepackage{verbdef}

\def\topfraction{.9}
\def\dbltopfraction{\topfraction}
\def\floatpagefraction{\topfraction}     % default .5
\def\dblfloatpagefraction{\topfraction}  % default .5
\def\textfraction{.1}

%HEVEA \footerfalse    % Disable hevea advertisement in footer

\newcommand{\code}[1]{\ifmmode{\mbox{\smaller\ttfamily{#1}}}\else{\smaller\ttfamily #1}\fi}
%% Hevea version omits "\smaller"
%HEVEA \renewcommand{\code}[1]{\ifmmode{\mbox{\ttfamily{#1}}}\else{\ttfamily #1}\fi}


% Add line between figure and text
\makeatletter
\def\topfigrule{\kern3\p@ \hrule \kern -3.4\p@} % the \hrule is .4pt high
\def\botfigrule{\kern-3\p@ \hrule \kern 2.6\p@} % the \hrule is .4pt high
\def\dblfigrule{\kern3\p@ \hrule \kern -3.4\p@} % the \hrule is .4pt high
\makeatother


\title{Annotation File Specification}
\author{\url{http://types.cs.washington.edu/annotation-file-utilities/} \\
\today}

\begin{document}

\maketitle

%HEVEA \setcounter{tocdepth}{2}
\tableofcontents

\section{Purpose:  External storage of annotations}

Java annotations are meta-data about Java program elements, as in
``\code{@Deprecated class Date \{ ... \}}''.
Ordinarily, Java annotations are written in the source code of a
\code{.java} Java source file.  When \code{javac} compiles the source code,
it inserts the annotations in the resulting \code{.class} file (as
``attributes'').

Sometimes, it is convenient to specify the annotations outside the source
code or the \code{.class} file.
\begin{itemize}
%BEGIN LATEX
\itemsep 0pt \parskip 0pt
%END LATEX
\item
  When source code is not available, a textual file provides a format for
  writing and storing annotations that is much easier to read and modify
  than a \code{.class} file.  Even if the eventual purpose is to insert the
  annotations in the \code{.class} file, the annotations must be specified
  in some textual format first.
\item
  Even when source code is available, sometimes it should not be changed,
  yet annotations must be stored somewhere for use by tools.
\item
  A textual file for annotations can eliminate code clutter.  A developer
  performing some specialized task (such as code verification,
  parallelization, etc.)\ can store annotations in an annotation file without
  changing the main version of the source code.  (The developer's private
  version of the code could contain the annotations, but the developer
  could copy them to the separate file before committing changes.)
\item
  Tool writers may find it more convenient to use a textual file, rather
  than writing a Java or \code{.class} file parser.
\item
  When debugging annotation-processing tools, a textual file format
  (extracted from the Java or \code{.class} files) is easier to read, and
  is easier for use in testing.
\end{itemize}

All of these uses require an external, textual file format for Java annotations.
The external file format should be easy for people to create, read, and
modify.
%
An ``annotation file'' serves this purpose by specifying a set of
Java annotations.
The Annotation File Utilities
(\url{http://types.cs.washington.edu/annotation-file-utilities/}) are a set
of tools that process annotation files.

The file format discussed in this document supports both standard Java SE 6
annotations and also the extended annotations proposed in JSR 308~\cite{JSR308}.
Section ``Class File Format Extensions'' of the JSR 308 design document
explains how the extended annotations are stored in the \code{.class}
file.  The
annotation file closely follows the class file format.
In that sense, the current design is extremely low-level, and users
probably would not want to write the files by hand (but might fill in a
template that a tool generated automatically).  As future work, we should
design a more
user-friendly format that permits Java signatures to be directly specified.
Furthermore, since the current design is closely aligned to the class file,
it is convenient for tools that operate on \code{.class} files but less
convenient for tools that operate on \code{.java} files.
For the short term, the low-level format will serve our purpose, which is
primarily to enable testing by the Javari developers.



%% I don't like this, as it may force distributing logically connected
%% elements all over a file system.  Users should be permitted, but not
%% forced, to adopt such a file structure. -MDE
%   Each file corresponds to exactly one
% ``.class'' file, so (for instance) inner classes are written in
% separate annotation files, named in the same ``{\tt
% OuterClass\$InnerClass}'' pattern as the ``.class'' file.


By convention, an annotation file ends with ``\code{.jaif}'' (for ``Java
annotation index file''), but this is not required.


% \verbdef\lineend|"\n"|

%BEGIN LATEX
\DefineShortVerb{\|}
\SaveVerb{newline}|\n|
\UndefineShortVerb{\|}
\newcommand{\lineend}{\bnflit{\UseVerb{newline}}}
%END LATEX
%HEVEA \newcommand{\bs}{\char"5C}
%HEVEA \newcommand{\lineend}{\bnflit{\bs{}n}}

% literal
\newcommand{\bnflit}[1]{\textrm{``}\textbf{#1}\textrm{''}}
% non-terminal
\newcommand{\bnfnt}[1]{\textsf{\emph{#1}}}
% comment
\newcommand{\bnfcmt}{\rm \# }
% alternative
\newcommand{\bnfor}{\ensuremath{|}}


\section{Grammar conventions}

% TODO: How is "[...]" different from "(...)?" ?  Why are two versions
% necessary? -MDE

Throughout this document, ``name'' is any valid Java simple name or 
fully qualified name, ``type'' is any valid type, and ``value'' is any
valid Java constant, and quoted strings are literal values.
%
The Kleene qualifiers ``*'' (zero or more), ``?'' (zero or one), and ``+''
(one or more) denote plurality of a grammar element.
%
A vertical bar (``\bnfor'') separates alternatives.
Parentheses (``()'') denote grouping, and square brackets (``[]'')
denote optional syntax, which is equivalent to ``( ... ) ?''.

In the annotation file, 
whitespace (excluding
newlines) is optional with one exception: no space is permitted
between an ``@'' character and a subsequent name. Indentation is
ignored, but is encouraged to maintain readability of the hierarchy of
program elements in the class (see the example in section
\ref{sec:example}).

Comments can be written throughout the annotation file using the double-slash
syntax employed by Java for single-line comments: anything following
two adjacent slashes (``//'') until the first newline is a comment.
This is omitted from the grammar for simplicity.
Block comments (``/* ... */'') are not allowed.




\subsection{Annotation file}

The annotation file itself contains one or more
package definitions; each package definition describes one or more
annotations and classes in that package.

\begin{tabbing}
\qquad \= \kill
\bnfnt{annotation-file} ::= \\
\qquad    \bnfnt{package-definition}+
\end{tabbing}

The annotation file may omit certain program elements --- for instance, it
may mention only some of the packages in your program, or only some of the
classes in a package, or only some of the fields or methods of a class.
Program elements that do not appear in the annotation file are treated as
unannotated.


\subsection{Package definitions}
\label{sec:package-definition}

Package definitions describe a package containing a list of annotation
definitions and classes.  A package definition also contains any
annotations on the package itself (such as those from a
\code{package-info.java} file).

\begin{tabbing}
\qquad \= \kill
\bnfnt{package-definition} ::= \\
\qquad    \bnfcmt{To specify the default package, omit the name.} \\
\qquad    \bnfcmt{Annotations on the default package are not allowed.} \\
\qquad    \bnflit{package} [ \bnfnt{name}? \bnflit{:} \bnfnt{annotation}* ] \lineend \\
\qquad    ( \bnfnt{annotation-definition} \bnfor{} \bnfnt{class-definition} ) *
\end{tabbing}


\subsection{Annotation definitions}
\label{sec:annotation-definition}

An annotation definition describes the annotation's fields and their
types, so that they may be referenced in a compact way throughout the
annotation file.

The Annotation File Utilities can read annotation definitions from the
classpath, so it is optional to define them in the annotation file.

If an annotation is defined in the annotation file, then it must be defined
before it is used.
% either on a program element or as a field of another annotation definition.
(This requirement makes it impossible to define, in an
annotation file, an annotation that is meta-annotated with itself.)
In the annotation file, the annotation definition appears within the
package that defines the annotation.  The annotation may be applied to
elements of any package.


\begin{tabbing}
\qquad \= \kill
\bnfnt{annotation-definition} ::= \\
\qquad    \bnflit{annotation} \bnflit{@}\bnfnt{name} \\
\qquad    [ \bnflit{:} \bnfnt{annotation-field-definition}+ ] \\
\qquad    \lineend \\
\\
\bnfnt{annotation-field-definition} ::= \\
\qquad    \bnfnt{type} \bnfnt{name} \lineend
\end{tabbing}

\subsection{Class definitions}

Class definitions describe the annotations present on the various
program elements.  It is organized according to the hierarchy of fields
and methods in the class.
Class definitions
are defined by the \code{class-definition} production of the following grammar.

Inner classes are treated as ordinary classes whose names happen to contain
\code{\$} signs and must be defined at the top level of a class definition file.
(To change this, the grammar would have to be extended with a closing
delimiter for classes; otherwise, it would be ambiguous whether a
field/method appearing after an inner class definition belonged to the
inner class or the outer class.)

% TODO: Are we sure about the comment for the "annotation" section below?
% Perhaps we could add that if a class is in the same package
% as an annotation it may always use the simple name (even if there's another
% annotation with the same simple name in another package)? - MP 06/28

% Currently the original examples do not have fully qualified names, Matt M.'s
% example does, and Matt M.'s parser requires fully qualified names.  Matt M.
% wants the specification to always require fully qualified names.  Matt P. says
% we'll make a final decision and reconcile the examples later. -- Matt M. 7/5

\begin{tabbing}
\qquad \= \kill
\bnfnt{annotation} ::= \\
\qquad    \bnfcmt The name may be the annotation's simple name, unless the file \\
\qquad    \bnfcmt contains definitions for two annotations with the same simple name. \\
\qquad    \bnfcmt In this case, the fully-qualified annotation name is required. \\
\qquad    \bnflit{@}\bnfnt{name} [ \bnflit{(} \bnfnt{annotation-field} [ \bnflit{,} \bnfnt{annotation-field} ]+ \bnflit{)} ] \\
\\
\bnfnt{annotation-field} ::= \\
\qquad    \bnfcmt In Java, if a single-field annotation has a field named \\
\qquad    \bnfcmt ``\texttt{value}'', and that field name may be elided in uses of the \\
\qquad    \bnfcmt annotation:   ``\texttt{@A(12)}'' rather than ``\texttt{@A(value=12)}''. \\
\qquad    \bnfcmt Furthermore

\qquad    \bnfcmt The same convention holdss in an annotation file. \\
\qquad    \bnfnt{name} \bnflit{=} \bnfnt{value} \\
\\
\bnfnt{class-definition} ::= \\
\qquad    \bnflit{class} \bnfnt{name} \bnflit{:} \bnfnt{annotation}* \lineend  \\
\qquad        \bnfnt{bound-definition}* \\
\qquad        \bnfnt{field-definition}*  \\
\qquad        \bnfnt{method-definition}* \\
\\
\bnfnt{field-definition} ::= \\
\qquad    \bnfcmt The annotation on the \bnlit{field} line is that of the field declaration, \\
\qquad    \bnfcmt while the annotation on the \bnlit{type} line is that of outermost type. \\
\qquad    \bnflit{field} \bnfnt{name} \bnflit{:} \bnfnt{annotation}* \lineend  \\
\qquad        \bnflit{type:} \bnfnt{annotation}* \lineend \\
\qquad        \bnfnt{type-argument-or-array-definition}* \\
\\
\bnfnt{method-definition} ::= \\
\qquad    \bnfcmt The method-key consists of the name followed by the signature \\
\qquad    \bnfcmt in JVML format, for example: ``\code{foo([ILjava/lang/String;)V}''. \\
\qquad    \bnfcmt The annotation on the \bnflit{method} line is that on the method not the return value. \\
\qquad    \bnflit{method} \bnfnt{method-key} \bnflit{:} \bnfnt{annotation}* \lineend \\
\qquad        \bnfnt{bound-definition}* \\
\qquad        \bnfnt{return-definition}? \\
\qquad        \bnfnt{receiver-definition}? \\
\qquad        \bnfnt{parameter-definition}* \\
\qquad        \bnfnt{variable-definition}* \\
\qquad        \bnfnt{typecast-definition}* \\
\qquad        \bnfnt{instanceof-definition}* \\
\qquad        \bnfnt{new-definition}* \\
\\
\bnfnt{type-argument-or-array-definition} ::= \\
\qquad    \bnfcmt The integer list here contains the values of the ``\ahref{http://types.cs.washington.edu/jsr308/specification/java-annotation-design.html#class-file:ext:ri:typearg}{location}'' array~\cite{JSR308}. \\ % Section 4.3.12
\qquad    \bnflit{inner-type} \bnfnt{integer} [ \bnflit{,} \bnfnt{integer} ]+ \bnflit{:} \bnfnt{annotation}* \lineend \\
\\
\bnfnt{bound-definition} ::= \\
\qquad    \bnfcmt The integers are respectively the parameter and bound indices of \\
\qquad    \bnfcmt the type parameter bound~\cite{JSR308}. \\
\qquad    \bnflit{bound} ,\bnfnt{integer} \&\bnfnt{integer} \bnflit{:} \bnfnt{annotation}* \lineend \\
\qquad    \bnfnt{type-argument-or-array-definition}* \\
\\
\bnfnt{return-definition} ::=  \\
\qquad    \bnflit{return:} \bnfnt{annotation}* \lineend \\
\qquad        \bnfnt{type-argument-or-array-definition}* \\
\\
\bnfnt{receiver-definition} ::=  \\
\qquad    \bnflit{receiver:} \bnfnt{annotation}* \lineend \\
\\
\bnfnt{parameter-definition} ::= \\
\qquad    \bnfcmt the integer is the index of the parameter in the method \\
\qquad    \bnfcmt (i.e., 0 is the first method parameter) \\
\qquad    \bnfcmt The annotation on the \bnlit{parameter} line is that of the parameter declaration, \\
\qquad    \bnfcmt while the annotation on the \bnlit{type} line is that of outermost type of the parameter. \\
\qquad    \bnflit{parameter} \bnfnt{integer} \bnflit{:} \bnfnt{annotation}* \lineend \\
\qquad        \bnflit{type:} \bnfnt{annotation}* \lineend \\
\qquad    \bnfnt{type-argument-or-array-definition}* \\
\\
\bnfnt{variable-definition} ::= \\
\qquad    \bnfcmt The integers are respectively the index, start, and length \\
\qquad    \bnfcmt fields of the annotations on this variable~\cite{JSR308}. \\
\qquad    \bnflit{local} \bnfnt{integer} \bnflit{\#} \bnfnt{integer} \bnflit{+} \bnfnt{integer} \bnflit{:} \bnfnt{annotation}* \lineend \\
\qquad    \bnfnt{type-argument-or-array-definition}* \\
\\
\bnfnt{typecast-definition} ::= \\
\qquad    \bnfcmt The integer is the offset field of the annotation~\cite{JSR308}. \\
\qquad    \bnflit{typecast} \bnflit{\#} \bnfnt{integer} \bnflit{:} \bnfnt{annotation}* \lineend \\
\qquad    \bnfnt{type-argument-or-array-definition}* \\
\\
\bnfnt{instanceof-definition} ::= \\
\qquad    \bnfcmt The integer is the offset field of the annotation~\cite{JSR308}. \\
\qquad    \bnflit{instanceof} \bnflit{\#} \bnfnt{integer} \bnflit{:} \bnfnt{annotation}* \lineend \\
\\
\bnfnt{new-definition} ::= \\
\qquad    \bnfcmt the integer is the offset field of the annotation~\cite{JSR308}. \\
\qquad    \bnflit{new} \bnflit{\#} \bnfnt{integer} \bnflit{:} \bnfnt{annotation}* \lineend  \\
\qquad    \bnfnt{type-argument-or-array-definition}*
\end{tabbing}


\subsection{Dependence on bytecode offsets}

For annotations on expressions (typecasts, instanceof, new, etc.), the
annotation file uses offsets into the bytecode array of the class file to
indicate the specific expression to which the annotation refers.  Because
different compilation strategies yield different \code{.class} files, a
tool that maps such annotations from an annotation file into source code must
have access to the specific \code{.class} file that was used to generate
the annotation file.  For non-expression annotations such as those on methods,
fields, classes, etc., the \code{.class} file is not necessary.


\section{Example}
\label{sec:example}

Consider the code of Figure~\ref{fig:java-example}.
Figure~\ref{fig:annotation-file-examples} shows two legal annotation files
each of which represents its annotations.


\begin{figure}
\begin{verbatim}
package p1;

import p2.*; // for the annotations @A through @D

public @A(12) class Foo {

    public int bar;             // no annotation
    private @B List<@C String> baz;

    public Foo(@B List<@C String> a) @D("spam") {
        @B List<@C String> l = new LinkedList<@C String>();
        l = (@B List<@C String>)l;
    }
}
\end{verbatim}
\caption{Example Java code with annotations.}
\label{fig:java-example}
\end{figure}


\begin{figure}
\begin{tabular}{|c|c|}
\hline
\begin{minipage}[t]{.5\textwidth}
\begin{verbatim}
package p2:
annotation @A:
    int value
annotation @B:
annotation @C:
annotation @D:
    String value

package p1:
class Foo: @A(value=12)

    field bar:

    field baz: @B
        inner-type 0: @C

    method <init>():
        parameter #0: @B
            inner-type 0: @C
        receiver: @D(value="spam")
        local 1 #3+5: @B
            inner-type 0: @C
        typecast #7: @B
            inner-type 0: @C
        new #0:
            inner-type 0: @C
\end{verbatim}
\end{minipage}
&
\begin{minipage}[t]{.45\textwidth}
\begin{verbatim}
package p2:
annotation @A
    int value

package p2:
annotation @B

package p2:
annotation @C

package p2:
annotation @D
    String value

package p1:
class Foo: @A(value=12)

package p1:
class Foo:
    field baz: @B

package p1:
class Foo:
    field baz:
        inner-type 0: @C

// ... definitions for p1.Foo.<init>()
// omitted for brevity
\end{verbatim}
\end{minipage}
\\
\hline
\end{tabular}

\caption{Two distinct annotation files each corresponding to the code of
  Figure~\ref{fig:java-example}.}
\label{fig:annotation-file-examples}
\end{figure}


\section{Types and values}
\label{sec:typesvalues}

The Java language permits several types for annotation fields: primitives,
\code{String}s, \code{java.lang.Class} tokens (possibly parameterized),
enumeration constants, subannotations, and one-dimensional arrays of these.

These \textbf{types} are represented in an annotation file as follows:
\begin{itemize}
\item Primitive: the name of the primitive type, such as \code{boolean}.
\item String: \code{String}.
\item Class token: \code{Class}; the parameterization, if any, is not
represented in annotation files.
\item Enumeration constant: \code{enum} followed by the fully qualified name of
the enumeration class, such as \code{enum java.lang.Thread\$State}.
\item Array: The representation of the element type followed by \code{[]}, such
as \code{String[]}, with one exception: an annotation definition may specify
a field type as \code{unknown[]} if, in all occurrences of that annotation in
the annotation file, the field value is a zero-length array.\footnotemark
\footnotetext{There is a design flaw in the format of array field values in a
class file.  An array does not itself specify an element type; instead, each
element specifies its type.  If the annotation type \code{X} has an array field
\code{arr} but \code{arr} is zero-length in every \code{@X} annotation in the
class file, there is no way to determine the element type of \code{arr} from the
class file.  This exception makes it possible to define \code{X} when the class
file is converted to an annotation file.}
\end{itemize}

Annotation field \textbf{values} are represented in an annotation file as follows:
\begin{itemize}
\item Numeric primitive value: literals as they would appear in Java source
code.
\item Boolean: \code{true} or \code{false}.
\item Character: A single character or escape sequence in single quotes, such
as \code{'A'} or \code{'\char`\\''}.
\item String: A string literal as it would appear in source code, such as
\code{"\char`\\"Yields falsehood when quined\char`\\" yields falsehood when quined."}.
\item Class token: The fully qualified name of the class (using \code{\$} for
inner classes) or the name of the primitive type or \code{void}, possibly
followed by \code{[]}s representing array layers, followed by \code{.class}.
Examples: \code{java.lang.Integer[].class}, \code{java.util.Map\$Entry.class},
and \code{int.class}.
\item Enumeration constant: the name of the enumeration constant, such as
\code{RUNNABLE}.
\item Array: a sequence of elements inside \code{\char`\{\char`\}} with a comma
between each pair of adjacent elements; a comma following the last element is
optional as in Java.  Also as in Java, the braces may be omitted if the
array has only one element.
Examples: \code{\char`\{1\char`\}}, \code{1},
\code{\char`\{true, false,\char`\}} and \code{\char`\{\char`\}}.
\end{itemize}

The following example annotation file shows how types and values are represented.

\begin{verbatim}
package p1:

annotation @ClassInfo:
    String remark
    Class favoriteClass
    Class favoriteCollection // it's probably Class<? extends Collection>
                             // in source, but no parameterization here
    char favoriteLetter
    boolean isBuggy
    enum p1.DebugCategory[] defaultDebugCategories
    @p1.CommitInfo lastCommit

annotation @CommitInfo:
    byte[] hashCode
    int unixTime
    String author
    String message

class Foo: @p1.ClassInfo(
    remark="Anything named \"Foo\" is bound to be good!",
    favoriteClass=java.lang.reflect.Proxy.class,
    favoriteCollection=java.util.LinkedHashSet.class,
    favoriteLetter='F',
    isBuggy=true,
    defaultDebugCategories={DEBUG_TRAVERSAL, DEBUG_STORES, DEBUG_IO},
    lastCommit=@p1.CommitInfo(
        hashCode={31, 41, 59, 26, 53, 58, 97, 92, 32, 38, 46, 26, 43, 38, 32, 79},
        unixTime=1152109350,
        author="Joe Programmer",
        message="First implementation of Foo"
    )
)
\end{verbatim}


\section{Alternative formats}

We mention two alternatives to the format described in this document.
Each of them has its own merits.
In the future, the other formats could be implemented, along with tools for
converting among them.
% Then, we can see which of the formats programmers prefer in practice.



An alternative to the format described in this document would be XML\@.  
% It would be easy to use an XML format to augment the one proposed here, but
XML does not seem to provide any compelling advantages.  Programmers
interact with annotation files in two ways:  textually (when reading, writing,
and editing annotation files) and programmatically (when writing
annotation-processing tools).  Textually, XML can be
very hard to read; style sheets mitigate this
problem, but editing XML files remains tedious and error-prone.
Programmatically, a layer of abstraction (an API) is needed in any event, so it
makes little difference what the underlying textual representation is.
XML files are easier to parse, but the parsing code only needs to be
written once and is abstracted away by an API to the data structure.


Another alternative is a format like the \code{.spec}/\code{.jml} files
of JML~\cite{LeavensBR2006:JML}.  The format is similar to Java code, but
all method bodies are empty, and users can annotate the public members of a
class.  This is easy for Java programmers to read and understand.  (It is a
bit more complex to implement, but that is not particularly germane.)
Because it does not permit complete specification of a class's annotations
(it does not permit annotation of method bodies), it is not appropriate for
certain tools, such as type inference tools.  However, it might be desirable
to adopt such a format for public members, and to use the format
described in this document primarily for method bodies.



\bibliographystyle{alpha}
\bibliography{bibstring-unabbrev,types,ernst,invariants,generals,alias}

\end{document}

% LocalWords:  java javac OuterClass InnerClass TODO Kleene MP subannotations
% LocalWords:  enum arr quined int pt instanceof RUNTIME JVML ILjava boolean
% LocalWords:  programmatically jml ernst jaif whitespace
